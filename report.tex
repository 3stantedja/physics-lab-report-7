\documentclass{article}
\usepackage[utf8]{inputenc}
\usepackage{tikz}
\usepackage{graphicx}

\usepackage{mhsetup}
\usepackage{mathtools}

\usepackage{amssymb}
\usepackage{textcomp}
\usepackage{siunitx}

\usepackage[parfill]{parskip}
\usepackage{array}

\sisetup{separate-uncertainty,multi-part-units=single}

\numberwithin{equation}{section}

\title{Experiment 7 — Standing waves in water}
\author{Laurence Amadeus Tristan, Nero Su, Weihong Deng}
\date{28/02/2019}

\begin{document}
\maketitle

\section{Purpose}
The purpose of this experiment is to figure out the speed of sound by examining the resonances of an open-ended resonance tube.

\section{Theory}
A standing wave consists two traveling waves interfering with each other in a particular manner that makes it appear stationary as a whole. As such, they consist of both nodes and antinodes. Nodes are positions where the particles in the air are stationary and not moving, whereas antinodes are the opposite of nodes, where the air particles moves back and forth at its greatest amplitude possible. One such example of a standing wave is resonance.

When resonance (i.e. standing waves) occurs, there will be a series of antinodes and nodes set in fixed locations along the air column.  This happens when the length of the air column \(L\) and the wavelength of the sound \(\lambda\) satisfies the equation:

\begin{equation} \label{eq:t1}
  L = N \ \frac{\lambda}{4}, \qquad N = \{1, 3, 5, \ldots\}
\end{equation}

\begin{tabbing}
  where \= \(L\) \= = length of air column, \\
  \> \(N\) \> = resonance order (odd integer values only, e.g. 1, 3, 5, \ldots), \\
  \> \(\lambda\) \> = wavelength of sound.
\end{tabbing}

Note that the length of the air column \(L\) does not equal to the length of the air chamber \(D\) in the tube, as the final antinode is located a distance \(x\) away from the open end of the tube. In terms of \(D\), this means that

\begin{equation} \label{eq:t2}
  L = D + x
\end{equation}

\begin{tabbing}
  where \= \(L\) \= = length of air column \\
  \> \(D\) \> = length of the air chamber, \\
  \> \(x\) \> = end correction.
\end{tabbing}

The end correction \(x\) is constant. This value depends on the diameter of the tube and the particular sound frequency that is made by the source, and cannot be dependent on the resonance order \(N\).

\section{Procedure}

\section{Data}
\begin{table}[!h]
  \centering
  \begin{tabular}{|c|l|l|}
    \hline
    Experiment no.  & Note value            & Frequency (\(f\))/\si{Hz} \\
    \hline
    1               & \(\mathrm{F}_4\)      & 349.2 \\
    2               & \(\sim \mathrm{G}_4\) & 392 \\
    3               & \(\sim \mathrm{A}_4\) & 486.7 \\
    4               & \(\mathrm{C}_5\)      & 523.2 \\
    \hline
  \end{tabular}
  \caption{Description of each experiment}
  \label{table:d1}
\end{table}

\begin{table}[!h]
  \centering
  \begin{tabular}{|l|l|l|}
    \hline
    Variable  & Value                       & Description\\
    \hline
    \(T_C\)   & \SI{24(1)}{\celsius}        & {Room temperature in Celsius}\\
    \(d\)     & \SI{3.366(2)}{\centi\metre} & {Inner diameter of the resonance tube}\\
    \hline
  \end{tabular}
  \caption{List of measured constant variables in the experiment}
  \label{table:d2}
\end{table}

\begin{table}[!h]
  \centering
  \begin{tabular}{|l|l|l|l|l|}
    \hline
    \(N\) & \(D_1 / \pm 1\) \si{cm} & \(D_2 / \pm 1\) \si{cm} & \(D_3 \pm 1\) \si{cm} & \(D_4 \pm 1\) \si{cm} \\
    \hline
    1  & 24  & 21  & 19  & 15 \\
    2  & 73  & 65  & 59  & 48 \\
    3  & --  & --  & --  & 81 \\
    \hline
  \end{tabular}
  \caption{Experimental data for all four experiments}
  \label{table:d3}
\end{table}

\section{Analysis}
First, combine equations \eqref{eq:t1} and \eqref{eq:t2} and solve for \(D\).
\begin{align} \label{eq:a1}
  \frac{N \lambda}{4} &= D + x \notag \\
  D &= - \left (\frac{N \lambda}{4} \right) + x \notag \\
  D &= N \left(\frac{\lambda}{4} \right) - x
\end{align}

As the wavelength \(\lambda\) is constant for each tuning fork, we can safely assume that the equation \eqref{eq:a1} is linear.

Now, graph the data for Experiment 4

\section{Discussion}


\end{document}
